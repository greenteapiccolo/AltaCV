%%%%%%%%%%%%%%%%%
% This is an sample CV template created using altacv.cls
% (v1.7, 9 August 2023) written by LianTze Lim (liantze@gmail.com). Compiles with pdfLaTeX, XeLaTeX and LuaLaTeX.
%
%% It may be distributed and/or modified under the
%% conditions of the LaTeX Project Public License, either version 1.3
%% of this license or (at your option) any later version.
%% The latest version of this license is in
%%    http://www.latex-project.org/lppl.txt
%% and version 1.3 or later is part of all distributions of LaTeX
%% version 2003/12/01 or later.
%%%%%%%%%%%%%%%%

%% Use the "normalphoto" option if you want a normal photo instead of cropped to a circle
% \documentclass[10pt,a4paper,normalphoto]{altacv}

\documentclass[10pt,a4paper,ragged2e,withhyper]{altacv}
%% AltaCV uses the fontawesome5 and packages.
%% See http://texdoc.net/pkg/fontawesome5 for full list of symbols.

% Change the page layout if you need to
\geometry{left=1.25cm,right=1.25cm,top=1.5cm,bottom=1.5cm,columnsep=1.2cm}

% The paracol package lets you typeset columns of text in parallel
\usepackage{paracol}

% Change the font if you want to, depending on whether
% you're using pdflatex or xelatex/lualatex
% WHEN COMPILING WITH XELATEX PLEASE USE
% xelatex -shell-escape -output-driver="xdvipdfmx -z 0" sample.tex
\ifxetexorluatex
  % If using xelatex or lualatex:
  \setmainfont{Roboto Slab}
  \setsansfont{Lato}
  \renewcommand{\familydefault}{\sfdefault}
\else
  % If using pdflatex:
  \usepackage[rm]{roboto}
  \usepackage[defaultsans]{lato}
  % \usepackage{sourcesanspro}
  \renewcommand{\familydefault}{\sfdefault}
\fi

% Change the colours if you want to
\definecolor{SlateGrey}{HTML}{2E2E2E}
\definecolor{LightGrey}{HTML}{666666}
\definecolor{DarkPastelRed}{HTML}{450808}
\definecolor{PastelRed}{HTML}{8F0D0D}
\definecolor{GoldenEarth}{HTML}{E7D192}
\colorlet{name}{black}
\colorlet{tagline}{PastelRed}
\colorlet{heading}{DarkPastelRed}
\colorlet{headingrule}{GoldenEarth}
\colorlet{subheading}{PastelRed}
\colorlet{accent}{PastelRed}
\colorlet{emphasis}{SlateGrey}
\colorlet{body}{LightGrey}

% Change some fonts, if necessary
\renewcommand{\namefont}{\Huge\rmfamily\bfseries}
\renewcommand{\personalinfofont}{\footnotesize}
\renewcommand{\cvsectionfont}{\LARGE\rmfamily\bfseries}
\renewcommand{\cvsubsectionfont}{\large\bfseries}


% Change the bullets for itemize and rating marker
% for \cvskill if you want to
\renewcommand{\cvItemMarker}{{\small\textbullet}}
\renewcommand{\cvRatingMarker}{\faCircle}
% ...and the markers for the date/location for \cvevent
% \renewcommand{\cvDateMarker}{\faCalendar*[regular]}
% \renewcommand{\cvLocationMarker}{\faMapMarker*}


% If your CV/résumé is in a language other than English,
% then you probably want to change these so that when you
% copy-paste from the PDF or run pdftotext, the location
% and date marker icons for \cvevent will paste as correct
% translations. For example Spanish:
% \renewcommand{\locationname}{Ubicación}
% \renewcommand{\datename}{Fecha}


%% Use (and optionally edit if necessary) this .tex if you
%% want to use an author-year reference style like APA(6)
%% for your publication list
% \input{pubs-authoryear.cfg}

%% Use (and optionally edit if necessary) this .tex if you
%% want an originally numerical reference style like IEEE
%% for your publication list
\usepackage[backend=biber,style=ieee,sorting=ydnt,defernumbers=true]{biblatex}
%% For removing numbering entirely when using a numeric style
\setlength{\bibhang}{1.25em}
\DeclareFieldFormat{labelnumberwidth}{\makebox[\bibhang][l]{\itemmarker}}
\setlength{\biblabelsep}{0pt}
\defbibheading{pubtype}{\cvsubsection{#1}}
\renewcommand{\bibsetup}{\vspace*{-\baselineskip}}
\AtEveryBibitem{%
  \iffieldundef{doi}{}{\clearfield{url}}%
}


%% sample.bib contains your publications
\addbibresource{sample.bib}

\begin{document}
\name{Jacques Chi}
\tagline{Software Engineer (4 years), full stack developer}
%% You can add multiple photos on the left or right
%\photoR{2.8cm}{Globe_High}
% \photoL{2.5cm}{Yacht_High,Suitcase_High}

\personalinfo{%
  % Not all of these are required!

  % TODO real data
  \email{xxxxxx@xxxxxx.com}
  \phone{000-00-0000}

  %\mailaddress{Åddrésş, Street, 00000 Cóuntry}
  \location{Montréal, CANADA}
  %\homepage{www.homepage.com}
  %\twitter{@twitterhandle}
  \linkedin{jacques-chi-500903621}
  \github{greenteapiccolo}
  %\orcid{0000-0000-0000-0000}
  %% You can add your own arbitrary detail with
  \printinfo{\faQuoteLeft}{Automate, Liberate!}
  % \printinfo{\faPaw}{Hey ho!}[https://example.com/]
  %% Or you can declare your own field with
  %% \NewInfoFiled{fieldname}{symbol}[optional hyperlink prefix] and use it:
  % \NewInfoField{gitlab}{\faGitlab}[https://gitlab.com/]
  % \gitlab{your_id}
  %%
  %% For services and platforms like Mastodon where there isn't a
  %% straightforward relation between the user ID/nickname and the hyperlink,
  %% you can use \printinfo directly e.g.
  % \printinfo{\faMastodon}{@username@instace}[https://instance.url/@username]
  %% But if you absolutely want to create new dedicated info fields for
  %% such platforms, then use \NewInfoField* with a star:
  % \NewInfoField*{mastodon}{\faMastodon}
  %% then you can use \mastodon, with TWO arguments where the 2nd argument is
  %% the full hyperlink.
  % \mastodon{@username@instance}{https://instance.url/@username}
}

\makecvheader
%% Depending on your tastes, you may want to make fonts of itemize environments slightly smaller
% \AtBeginEnvironment{itemize}{\small}

%% Set the left/right column width ratio to 6:4.
\columnratio{0.75}

% Start a 2-column paracol. Both the left and right columns will automatically
% break across pages if things get too long.
\begin{paracol}{2}

\cvsection{Experience}
\cvevent{Full stack engineer}{Société Générale}{01-2022 -- Ongoing}{Montréal, Canada}
\begin{itemize}
\item Work on multiple apps based on \textbf{React.js with TypeScript, react-query and bootstrap design system, Java Spring and PostgreSQL}, deployed with JDK pipeline
\item Developed monitoring application (code quality, library versions..) to improve business application health based on \textbf{React.js + Vite}
\item Work on code testing introduction to the team, with \textbf{react-testing-library (UI) and JUnit (API)}
\item Organized regular team meetings focused on \textbf{knowledge sharing, technical workshops, and collaborative problem-solving sessions}
\item \textbf{Technical ownership on multiple projects, including some architecture design like hexagonal}, work planning and business synchronization
\end{itemize}
\divider
\cvevent{Front end developer}{triPica}{10-2019 -- 11-2021}{Paris, France}
\begin{itemize}
\item Developed mobile applications with \textbf{React Native, deployed via Bitrise (cf Kaizen Mobile App on stores), project structure in monorepo}
\item Collaboration with \textbf{UI/UX designers based on material-ui}
\item Integrated \textbf{eSIM Carrier App along with Thales Digital Identity KYC native library (Android and iOS)} on mobile applications
\item Worked on \textbf{Stripe 3DSV2 payment system integration} on mobile applications
\item Worked on \textbf{in-app browser of mobile applications with react-native-web and configuration of Webpack}
\item Worked on mobile applications \textbf{push notification} system
\item Developed CRM \textbf{web applications with react-admin and redux, project structure in semi-module repository style}
\item Worked on business related \textbf{template web editor with Svelte}  
\end{itemize}
\divider
\cvevent{Internship - Front end developer}{IBM}{01-2019 -- 06-2019}{Paris, France}
Developed \textbf{modular dashboard web application with react-grid-layout and mobX}

\newpage

\cvsection{Projects}
\cvevent{Virtual / Augmented reality trend study}{University of Neveda, Las Vegas}{05-2018 -- 07-2018}{Las Vegas, USA}
Published a scientific paper and \textbf{presentation in Munich for AR \& VR Conference 2019}
\smallbreak
\divider
\cvevent{Pandemic handling with connected tags}{ESIEE Paris}{2018}{Paris, France}
Developed \textbf{bluetooth modules wristbands to simulate patients localization with FLIR thermal scan integration}
\smallbreak
\divider
\cvevent{French Robotic Cup}{ESIEE Paris}{2018}{Paris, France}
Developed \textbf{robot's collision avoidance with MbedOS + Python and a Raspberry Pi}
\smallbreak
\divider
\cvevent{Quality of Experience metrics measurement}{Laboratoire LISSI}{03-2016 -- 06-2016}{Vitry-sur-Seine, France}
Worked on \textbf{adaptive video streaming with DASH.js and interview-based qualitative study}

\medskip

\cvsection{Publications}
\printinfo{\faUsers}{Conference Proceedings}
\smallbreak
\textbf{AR \& VR Conference 2019 in Munich, Germany}
\smallbreak
\smallbreak
\printinfo{\faFile*[regular]}{Citation}
\smallbreak
\textbf{Augmented Reality and Virtual Reality}
\smallbreak
\url{https://doi.org/10.1007/978-3-030-37869-1}
%% Specify your last name(s) and first name(s) as given in the .bib to automatically bold your own name in the publications list.
%% One caveat: You need to write \bibnamedelima where there's a space in your name for this to work properly; or write \bibnamedelimi if you use initials in the .bib
%% You can specify multiple names, especially if you have changed your name or if you need to highlight multiple authors.
%\mynames{Lim/Lian\bibnamedelima Tze,
%  Wong/Lian\bibnamedelima Tze,
%  Lim/Tracy,
%  Lim/L.\bibnamedelimi T.}
%% MAKE SURE THERE IS NO SPACE AFTER THE FINAL NAME IN YOUR \mynames LIST
%\nocite{*}
%\printbibliography[heading=pubtype,title={\printinfo{\faBook}{Books}},type=book]

%divider

%\printbibliography[heading=pubtype,title={\printinfo{\faFile*[regular]}{Journal Articles}},type=article]

%\divider

%\printbibliography[heading=pubtype,title={\printinfo{\faUsers}{Conference Proceedings}},type=inproceedings]

\medskip

\cvsection{A Day of My Life}
% Adapted from @Jake's answer from http://tex.stackexchange.com/a/82729/226
% \wheelchart{outer radius}{inner radius}{
% comma-separated list of value/text width/color/detail}
\wheelchart{1.5cm}{0.5cm}{%
  1/8em/accent!20/Guqin instrument,
  2/8em/accent!40/Badminton,
  2/10em/accent!40/Baby Foot,
  4/8em/accent!80/{Dragon boating},
  2/6em/accent!60/Geek and random stuff,
  4/6em/accent!80/Automation
}

%% Switch to the right column. This will now automatically move to the second
%% page if the content is too long.
\switchcolumn

\cvsection{Strengths}
\cvtag{React.js}
\cvtag{Node.js}\\
\cvtag{Java Spring}
\cvtag{Python}
\smallbreak
\cvtag{Bash}
\cvtag{Git}
\cvtag{Kubernetes}\\
\cvtag{TypeScript}
\cvtag{Svelte}

%\cvsection{My Life Philosophy}

%\begin{quote}
%``Something smart or heartfelt, preferably in one sentence.''
%\end{quote}

%\cvsection{Most Proud of}

%\cvachievement{\faTrophy}{Fantastic Achievement}{and some details about it}

%\divider

%\cvachievement{\faHeartbeat}{Another achievement}{more details about it of course}

%\divider

%\cvachievement{\faHeartbeat}{Another achievement}{more details about it of course}

\cvsection{Languages}
\cvskill{French}{5}
\textit{Voltaire 765}
\smallbreak
\cvskill{English}{4}
\textit{TOEIC 980}
\smallbreak
\cvskill{Chinese}{2.5}
\smallbreak
\cvskill{Japanese}{1.5}
\smallbreak
\cvskill{Spanish}{1}

%% Yeah I didn't spend too much time making all the
%% spacing consistent... sorry. Use \smallskip, \medskip,
%% \bigskip, \vspace etc to make adjustments.

\cvsection{Education}
\cvevent{Engineering School}{ESIEE Paris}{2016 -- 2019}{}
Data, Networking and Internet of Things with merit
\smallbreak
\divider
\cvevent{Tech University}{University of Technology of Créteil-Vitry}{2014 -- 2016}{}
Networking and Telecommunication diploma with merit
\smallbreak
\divider
\cvevent{High School}{Saint-Exupéry High School}{2011 -- 2014}{}
Scientific diploma with merit (engineering specialization)

\newpage

\cvsection{Certifications}
\cvsubsection{Cisco, 2015 -- 2016}
\begin{itemize}
  \item CCNA 1: Introduction to Networks
  \item CCNA 3: Scaling Networks
\end{itemize}

% \divider

%\cvsection{Referees}

% \cvref{name}{email}{mailing address}
%\cvref{Prof.\ Alpha Beta}{Institute}{a.beta@university.edu}
%{Address Line 1\\Address line 2}

%\divider

%\cvref{Prof.\ Gamma Delta}{Institute}{g.delta@university.edu}
%{Address Line 1\\Address line 2}


\end{paracol}


\end{document}
