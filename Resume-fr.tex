\documentclass[10pt,a4paper,ragged2e,withhyper]{altacv}
\geometry{left=1.25cm,right=1.25cm,top=1cm,bottom=1.5cm,columnsep=1.2cm}

% The paracol package lets you typeset columns of text in parallel
\usepackage{paracol}

% Change the font if you want to, depending on whether
% you're using pdflatex or xelatex/lualatex
% WHEN COMPILING WITH XELATEX PLEASE USE
% xelatex -shell-escape -output-driver="xdvipdfmx -z 0" sample.tex
\ifxetexorluatex
  % If using xelatex or lualatex:
  \setmainfont{Roboto Slab}
  \setsansfont{Lato}
  \renewcommand{\familydefault}{\sfdefault}
\else
  % If using pdflatex:
  \usepackage[rm]{roboto}
  \usepackage[defaultsans]{lato}
  \usepackage{hyperref}
  % \usepackage{sourcesanspro}
  \renewcommand{\familydefault}{\sfdefault}
\fi

% Change the colours if you want to
\definecolor{SlateGrey}{HTML}{2E2E2E}
\definecolor{LightGrey}{HTML}{666666}
\definecolor{DarkPastelRed}{HTML}{450808}
\definecolor{PastelRed}{HTML}{8F0D0D}
\definecolor{GoldenEarth}{HTML}{E7D192}
\colorlet{name}{black}
\colorlet{tagline}{PastelRed}
\colorlet{heading}{DarkPastelRed}
\colorlet{headingrule}{GoldenEarth}
\colorlet{subheading}{PastelRed}
\colorlet{accent}{PastelRed}
\colorlet{emphasis}{SlateGrey}
\colorlet{body}{LightGrey}

% Change some fonts, if necessary
\renewcommand{\namefont}{\Huge\rmfamily\bfseries}
\renewcommand{\personalinfofont}{\footnotesize}
\renewcommand{\cvsectionfont}{\LARGE\rmfamily\bfseries}
\renewcommand{\cvsubsectionfont}{\large\bfseries}


% Change the bullets for itemize and rating marker
% for \cvskill if you want to
\renewcommand{\cvItemMarker}{{\small\textbullet}}
\renewcommand{\cvRatingMarker}{\faCircle}
% ...and the markers for the date/location for \cvevent
% \renewcommand{\cvDateMarker}{\faCalendar*[regular]}
% \renewcommand{\cvLocationMarker}{\faMapMarker*}


% If your CV/résumé is in a language other than English,
% then you probably want to change these so that when you
% copy-paste from the PDF or run pdftotext, the location
% and date marker icons for \cvevent will paste as correct
% translations. For example Spanish:
% \renewcommand{\locationname}{Ubicación}
% \renewcommand{\datename}{Fecha}

%% Use (and optionally edit if necessary) this .tex if you
%% want an originally numerical reference style like IEEE
%% for your publication list
\usepackage[backend=biber,style=ieee,sorting=ydnt,defernumbers=true]{biblatex}
%% For removing numbering entirely when using a numeric style
\setlength{\bibhang}{1.25em}
\DeclareFieldFormat{labelnumberwidth}{\makebox[\bibhang][l]{\itemmarker}}
\setlength{\biblabelsep}{0pt}
\defbibheading{pubtype}{\cvsubsection{#1}}
\renewcommand{\bibsetup}{\vspace*{-\baselineskip}}
\AtEveryBibitem{%
  \iffieldundef{doi}{}{\clearfield{url}}%
}


\AtBeginEnvironment{itemize}{\small}

\begin{document}

\name{Jacques Chi}
\tagline{Développeur informatique mid-senior, full-stack}
%% You can add multiple photos on the left or right
%\photoR{2.8cm}{Globe_High}
% \photoL{2.5cm}{Yacht_High,Suitcase_High}

\personalinfo{
  % TODO real data
  \email{}
  \phone{}
  \location{Montréal, CANADA}
  \linkedin{jacques-chi-500903621}
  \github{greenteapiccolo}
  \printinfo{\faThumbsUp}{Références (LinkedIn)}[https://www.linkedin.com/in/jacques-chi-500903621/details/recommendations/]
\quote
}

\makecvheader

\cvsection{Principalement sur..}
\cvtag{React.js}
\cvtag{Node.js}
\cvtag{Java Spring}
\cvtag{Python}
\cvtag{Svelte}
\cvtag{TypeScript}
\cvtag{Bash}
\cvtag{GitHub}

\cvsection{Expérience}
\cvevent{\textbf{Ingénieur Full Stack}}{Société Générale}{01-2022 -- 12-2023}{Montréal, Canada}
\vspace{-.5\baselineskip}
\begin{minipage}[t]{0.45\textwidth}
  \vspace{0pt}
  Projets (React.js et Java Spring Boot):
  \begin{itemize}
    \item Tableau de bord KPI avec des modules graphiques
    \item Système de reporting avec des services auto-planifiés
    \item Outils liés à l'ITSM (planificateur, détails, recherche, versioning) en format web
    \item Outil web Break Glass avec gestion de droits
    \item Tour de contrôle des applications, avec l'aide de GraphQL
  \end{itemize}
\end{minipage}
\hfill
\begin{minipage}[t]{0.45\textwidth}
  \vspace{0pt}
  Technique:
  \begin{itemize}
    \item Utilisation de l'architecture hexagonale avec des design patterns associés et autres
    \item Tests unitaires et d'intégration avec react-testing-library (UI) et JUnit (API)
    \item Utilisation de plusieurs méthodes de cache et de threading pour les performances côté UI et API
    \item Scripts Python planifiés avec Kubernetes Cron
    \item Surveillance des performances applicatives avec Kibana
  \end{itemize}
\end{minipage}

\medskip
Autres compétences:
\begin{itemize}
  \item Livraison de POC incluant l'architecture, la collaboration, la planification et son développement
  \item Organisation de connaissances, ateliers techniques et séances de résolution de problèmes collaboratives
  \item Owner technique sur plusieurs projets
  \item Mentorat sur certains aspects techniques
  \item Obtention de la 2ème place au America Region Secure Coding Challenge
  \item Structure d'équipe FTS
  \item Gestion de support et de feedback
\end{itemize}
\begin{itshape}
  \underline{Nouveaux outils:}
  \small{PostgreSQL, Vite, bootstrap etc.}
\end{itshape}

\divider

\cvevent{\textbf{Développeur front-end}}{triPica}{10-2019 -- 11-2021}{Paris, France}
\vspace{-.5\baselineskip}
\begin{minipage}[t]{0.45\textwidth}
  \vspace{0pt}
  Projets (React.js et React Native):
  \begin{itemize}
    \item Application hybride B2C React Native pour iOS et Android
    \item B2B CRM en TypeScript React.js
    \item Utilisation de l'internationalisation i18n centralisée pour toutes les applications
    \item Travail sur les notifications push des applications mobiles
    \item Éditeur de template web avec Svelte
    \item Intégration du système de paiement Stripe 3DSV2
    \item eKYC mobile avec les librairies natives Thalès
  \end{itemize}
\end{minipage}
\hfill
\begin{minipage}[t]{0.45\textwidth}
  \vspace{0pt}
  Technique:
  \begin{itemize}
    \item Projets en conception monorepo avec certains patterns (Factory, State...)
    \item Utilisation de Redux.js et de Normalization pour gérer des données UI plus importantes
    \item Test de l'application web avec snapshots et Storybook
    \item Débogage de bibliothèques externes et correction si nécessaire
  \end{itemize}
\end{minipage}

\bigskip
Autres compétences:
\begin{itemize}
  \item Applications B2C ayant 99\% de uptime avec des restrictions et deadlines externes et dette technique
  \item Collaboration avec des concepteurs UI/UX et des équipes externes
  \item Ownership de certaines fonctionnalités des applications
\end{itemize}
\begin{itshape}
  \underline{Nouveaux outils:}
  \small{material-ui, Bitrise, Jenkins, CircleCI etc.}
\end{itshape}

\divider

\cvevent{\textbf{Stage - développeur front-end}}{IBM}{01-2019 -- 07-2019}{Paris, France}
Projet: application web de tableau de bord modulaire KPI avec react-grid-layout, mobX et d3.js

Notes: auto-didacte; autonome; projet vivant

\newpage
\columnratio{0.7}
% Commencer un paracolomne en 2 colonnes. Les colonnes gauche et droite passeront automatiquement aux pages suivantes si elles sont trop longues
\begin{paracol}{2}

\cvsection{Projets}
\cvevent{Étude de tendance sur la réalité virtuelle / augmentée}{University of Neveda, Las Vegas}{05-2018 -- 07-2018}{Las Vegas, USA}
Publication d'un article scientifique et \textbf{présentation à Munich pour la conférence AR \& VR 2019}
\smallbreak
\divider
\cvevent{Gestion de pandémie avec tags connectés}{ESIEE Paris}{2018}{Paris, France}
Développement de \textbf{modules bluetooth pour bracelets afin de simuler la localisation des patients avec intégration de scanner thermique FLIR}
\smallbreak
\divider
\cvevent{Coupe de France de robotique}{ESIEE Paris}{2018}{Paris, France}
Développement de l'\textbf{évitement de collision du robot avec MbedOS + Python et un Raspberry Pi}
\smallbreak
\divider
\cvevent{Mesure des métriques de qualité d'expérience}{Laboratoire LISSI}{03-2016 -- 06-2016}{Vitry-sur-Seine, France}
Travail sur \textbf{le streaming vidéo adaptatif avec DASH.js basé sur des études via interview}

\medskip
\cvsection{Publications}
\printinfo{\faUsers}{Actes de conférence}
\smallbreak
\textbf{Conférence AR \& VR 2019 à Munich, Allemagne}
\smallbreak
\smallbreak
\printinfo{\faFile*[regular]}{Citations}
\smallbreak
\href{https://doi.org/10.1007/978-3-030-37869-1}{\textbf{Augmented Reality and Virtual Reality}}
\smallbreak
\href{https://www.cogitatiopress.com/mediaandcommunication/article/view/5316/2797}{\textbf{Methodological Reflections on Capturing Augmented Space: Insights From an Augmented Reality Field Study}}

\cvsection{Une journée de ma vie}
% Adapté de la réponse de @Jake sur http://tex.stackexchange.com/a/82729/226
% \wheelchart{rayon extérieur}{rayon intérieur}{
% liste séparée par des virgules de valeur/largeur du texte/couleur/détail}
\wheelchart{1.5cm}{0.5cm}{%
  1/10em/accent!30/Instrument Guqin,
  1.5/10em/accent!45/Badminton,
  2/10em/accent!60/Baby Foot,
  2.5/10em/accent!75/undefined,
  2.5/10em/accent!75/Geek et autre,
  1.5/10em/accent!45/Bateau dragon
}

%% Changer pour la colonne de droite. Cela passera automatiquement à la deuxième
%% page si le contenu est trop long.
\switchcolumn
\newpage

\cvsection{Langues}
\cvskill{Français}{5}
\textit{Voltaire 765}
\smallbreak
\cvskill{Anglais}{4.5}
\textit{TOEIC 980}
\smallbreak
\cvskill{Chinois}{2.5}
\smallbreak
\cvskill{Japonais}{1.5}
\smallbreak
\cvskill{Espagnol}{1}

\cvsection{Education}
\cvevent{École d'ingénieur}{ESIEE Paris}{2016 -- 2019}{}
Data, Réseaux et IoT avec mention
\smallbreak
\divider
\cvevent{Université Technologique}{Université de Technologie de Créteil-Vitry}{2014 -- 2016}{}
Diplôme de Réseaux et Télécom avec mention
\smallbreak
\divider
\cvevent{Lycée}{Lycée Saint-Exupéry}{2011 -- 2014}{}
Diplôme scientifique avec mention bien (spécialisation en ingénierie)

\cvsection{Certifications}
\cvsubsection{Cisco, 2015 -- 2016}
\begin{itemize}
  \item CCNA 1: Introduction to Networks
  \item CCNA 3: Scaling Networks
\end{itemize}

\end{paracol}
\end{document}
